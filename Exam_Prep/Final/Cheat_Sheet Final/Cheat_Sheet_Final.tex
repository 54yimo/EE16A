\documentclass{article}
\usepackage[left=3cm, right=3cm, top=3cm]{geometry}
\usepackage{amssymb}
\usepackage{amsmath}
\usepackage{color}
\usepackage{graphicx}
\begin{document}

\noindent{\Large Circuit Analysis}
\begin{itemize}
	\item KCL: The net current flowing out of (= into) any junction of a circuit is 0.
	\item KVL: $\sum\limits_{loop} V_k = 0$
	\item Ohm's Law: $V_{elem} = I_{elem}R$ (Volt = Ampere $\cdot$ Ohm $\Omega$)
	\item Passive sign convention: Positive current should enter the positive terminal and exit the negative terminal of an element. Positive power means that power is dissipated, and negative power means that power is being generated.
	\item Resistors: in series $R_{eq} = R_1+R_2$; in parallel $R_{eq} = R_1||R_2 = \frac{R_1R_2}{R_1+R_2}$
	% \item Steps for solving circuits
	% \begin{enumerate}
	% 	\item Pick a node and label it as ``ground''.
	% 	\item Label all remaining nodes as some $u_i$, representing the voltage relative to the ground node.
	% 	\item Label the current through every {\color{blue} non-wire element} (= elements below) in the circuit $i_n$.
	% 	\item Add $+/-$ labels (direction of voltage meas.) on each element by following the {\color{red} passive sign convention}.
	% 	\item Set up $A\vec{x}=\vec{b}$, where $\vec{x}$ is comprised of the $u_i$’s and $i_n$’s defined in the previous steps.
	% 	\item If there are $n$ nodes (including the ground node), use KCL on $(n-1)$ nodes to fill in $(n-1)$ rows of $A$ and $\vec{b}$.
	% 	\item If there are $m$ elements, use the $IV$ relationship of each element to fill in the remaining $m$ equations (rows of $A$ and values of $\vec{b}$).
	% 	\item Solve with linear algebra!
	% \end{enumerate}
	\item {\color{red} Voltage divider}: $V_{out} = \frac{R_{out}}{R_1+R_{out}}\cdot V_s$
	%\item Current divider: $R_1,R_2$ in parallel with independent current source $I_s$, then $I_{R_2} = \frac{R_1}{R_1+R_2}I_s = \frac{R_1||R_2}{R_2}I_s$
	\item $I = \frac{dQ}{dt}$ (Coulombs/second)
	\item {\color{red} $R = \rho\cdot\frac{L}{A}$}
	\item $P = VI = I^2R = \frac{V^2}{R} = \frac{dE}{dt} = V\frac{dQ}{dt}$
\end{itemize}

\pagebreak
\noindent{\Large Superposition and Equivalence}
\begin{itemize}
	\item Superposition: Zero all but 1 independent source (Voltage source: replace with a wire; Current source: replace with an open circuit)
	\item Def 15.1 (equivalent circuit): If we pick two terminals within a circuit, we say that another circuit is equivalent to the original circuit if it exhibits the same $I-V$ relationship at those two terminals. (In short, Two circuits are equivalent if they have the same $I-V$ relationship.)
	\item Thevenin equivalent circuit: 1 voltage source and 1 resistor (``in series'')
	\begin{itemize}
		\item Find $V_{Th}$: Connect an open circuit across the two output terminals and measure the voltage across them. This measured $V_{OC}$ equals $V_{Th}$.
	\end{itemize}
	\item Norton equivalent circuit: 1 current source and 1 resistor (``in parallel'')
	\begin{itemize}
		\item Find $I_{No}$: Connect a short circuit across the two output terminals and measure the current through it. This measured $I_{SC}$ equals $I_{No}$.
	\end{itemize}
	\item For $R_{eq}$ between nodes $a,b$:
	\begin{enumerate}
		\item Calculate $V_{th}$ and $I_{no}$ (superposition if necessary), and then $R_{eq} = \frac{V_{th}}{I_{no}}$ \\
		{\color{blue} N.B.} Works only if $\exists$ at least one independent source in the circuit. Else, $V_{th} = I_{no} = 0$.
		\item Zero out {\color{red} all} independent sources and apply a $V_{test}$ or $I_{test}$ to calculate the resulting $I_{test}$ or $V_{test}$ respectively. $R_{eq} = \frac{V_{test}}{I_{test}}$. \\
		{\color{blue} N.B.} Works for any circuit. When in doubt, use this.
	\end{enumerate}
	\item The power dissipated by the source in the original circuit is not the same as the power dissipated in the Thevenin equiv. circuit, but the power through $R_{load}$ is the same! Thevenin equiv. can be used to calculate the power through elements that are not part of the circuit that was transformed.
	\item Extra: {\color{red} $R_{Th} = R_{No}$}; Norton $\rightarrow$ Thevenin: $V_{th} = I_{No}R_{No}$; Thevenin $\rightarrow$ Norton: $I_{No} = \frac{V_{Th}}{R_{Th}}$;
	\item Extra: Resistors in parallel have an equivalent resistance $<$ any of the individual resistors (positive resistances).
	\item Extra: The power generated by the Thevenin equivalent circuit $\neq$ total power generated in the original circuit.
\end{itemize}

\pagebreak
\noindent{\Large Capacitors}
\begin{itemize}
	\item $C = I\cdot t = \frac{P}{V}t = \frac{Energy}{V}$
	\item $Q = CV_c\iff C = \frac{Q}{V_c}$ unit is Farad (F), $1A\cdot 1s$
	\item $I = \frac{dQ}{dt} = C\frac{dV_c}{dt}$ (general), so current is flowing through the capacitor $\iff$ the voltage across the capacitor is changing with time. ({\color{red} $I_c(t) = C\frac{dV_c(t)}{dt}$})
	\item $\Longrightarrow V_c(t) = \frac{{\color{red} I}}{C}t + V_c(0)$ Only valid when the current is constant over time.
	\item $C_{eq} = \frac{I_{test}}{\frac{dV_{test}}{dt}}$
	\item Capacitors: {\color{red} in series} $C_{eq} = C_1||C_2 = \frac{C_1C_2}{C_1+C_2}$; in parallel $C_{eq} = C_1+C_2$
	\item {\color{red} $C = \epsilon\frac{A}{d}$}
	\item Energy stored in capacitor: $E = \frac{1}{2} C_{eq}V^2$
	\item For $V_c(0) = 0V,$ then $V = \frac{I_st}{C}$.	If we measure the voltage at a known time $t$, we can solve for the capacitance. However, as time continues to pass, the voltage across the capacitor (and also the charge stored on the capacitor) will grow to $\infty$. It is very challenging to build an ideal current source that works over this large range of voltages, so our model quickly becomes unrealistic $\Longrightarrow$ periodic current source.
	\item Consider a square current $I_s$
	\begin{itemize}
		\item $I_s$ is constant from $t=0$ to $t = \frac{\tau}{2}$, so $V_c(t) = \frac{I_s}{C}t$ when $0\leq t\leq\frac{\tau}{2}$
		\item Similarly, $V_c(t) = -\frac{I_s}{C}(t-\frac{\tau}{2}) + \frac{I_s\tau}{2C}$ when $\frac{\tau}{2}<t<\tau$
	\end{itemize}

% \begin{figure} [h!]
% 	\begin{center}
% 	\includegraphics[width=0.5\linewidth]{/Users/Alan/Desktop/Periodic}
% 	\caption{Periodic Current Source}
% 	\label{fig}
% 	\end{center}
% \end{figure}

\end{itemize}

\pagebreak
\noindent{\Large Op-Amp and Golden Rules}
\begin{itemize}
	\item Apply KCL, then use Golden Rule(s) to find desired relationships.
	\item Op-amp (operational amplifier): $A>1$ is the gain. Op-amp acts $\sim$ a comparator since $A$ is very large.
% \begin{figure} [h!]
% 	\begin{center}
% 	\includegraphics[width=0.5\linewidth]{/Users/Alan/Desktop/op-amp}
% 	\caption{Op-amp Voltage Formula}
% 	\label{fig}
% 	\end{center}
% \end{figure}

	\item For an ideal op-amp, we have $A\rightarrow\infty$, which implies {\color{red} Golden Rules}:
	\begin{enumerate}
		\item $I_+ = I_- = 0$ holds regardless of whether there's negative feedback or not.
		\item $U_+ = U_-$ holds $\iff$ there's negative feedback.
	\end{enumerate}
	\item Checking if an op-amp is in {\color{blue} negative feedback}:
	\begin{enumerate}
		{\color{red}\item Zero out all independent sources} like we did in Thevenin-Norton Equivalences.
		\item Increase the output. If the feedback coming from the circuit as a result of increasing the output is in the opposite direction (i.e. decreasing), then the circuit is in negative feedback. If not, the circuit is in positive feedback or zero feedback.
	\end{enumerate}

	% \begin{figure} [h!]
	% \begin{center}
	% \includegraphics[width=0.3\linewidth]{/Users/Alan/Desktop/opamp-neg}
	% \caption{Negative Feedback}
	% \label{fig}
	% \end{center}
	% \end{figure}
	\item In this negative feedback setup, {\color{red} $V_{out} = V_{in} (1+\frac{R_1}{R_2})$}. Thus, As long as the two resistors are produced with the same error rate $\epsilon$, i.e., they have resistance $(1+\epsilon)R_1$ and $(1+\epsilon)R_2$, then the ratio between their resistance will remain the same, so $\frac{V_{out}}{V_{in}}$ is the same.

% \begin{figure} [h!]
% 	\begin{center}
% 	\includegraphics[width=0.3\linewidth]{/Users/Alan/Desktop/opamp-invert}
% 	\caption{Inverting Amplifier}
% 	\label{fig}
% 	\end{center}
% \end{figure}
	\item Inverting an op-amp gives {\color{red} $V_{out} = -\frac{R_f}{R_{in}} V_{in}$}. Here, gain $A = -\frac{R_f}{R_{in}}$ can have be any negative value.

% \begin{figure} [h!]
% 	\begin{center}
% 	\includegraphics[width=0.5\linewidth]{/Users/Alan/Desktop/opamp-neuron}
% 	\caption{Inverting Summing Amplifier}
% 	\label{fig}
% 	\end{center}
% \end{figure}
	\item This models an artificial neuron, withs $V_{ana} = -\frac{R_3}{R_1}V_1 - \frac{R_3}{R_2}V_2$, and so {\color{red} $V_{out} = -\frac{R_4}{R_4}V_{ana} = \frac{R_3}{R_1}V_1 + \frac{R_3}{R_2}V_2$}

% \begin{figure} [h!]
% 	\begin{center}
% 	\includegraphics[width=0.5\linewidth]{/Users/Alan/Desktop/opamp-unity}
% 	\caption{Unity Gain Buffer}
% 	\label{fig}
% 	\end{center}
% \end{figure}
	\item {\color{blue} Unity gain buffer} to cancel out the effects of loading, allowing $V_{speaker} = V_{DAC}\frac{A}{1+A}\sim V_{DAC}$ as $A\rightarrow\infty$.
	\item Buffers are a powerful tool because they allow us to split circuits into blocks that we can analyze separately and then combine later. When circuit blocks behave the same way regardless of what they’re are connected to, we don’t need to worry about what’s inside, making it much easier to design complex circuits.

	\item Extra: An op-amp cannot operate without externally supplied power. Any power an op-amp delivers to a load comes from the supply voltages since it can't generate its own energy.
	\item Extra: Since an op-amp has infinite input resistance, there's no current flowing into the input terminals, so it doesn't change the voltage of any circuit it's connected to.

% \begin{figure} [h!]
% 	\begin{center}
% 	\includegraphics[width=0.5\linewidth]{/Users/Alan/Desktop/invert-or-not}
% 	\caption{Inverting v. Non-Inverting}
% 	\label{fig}
% 	\end{center}
% \end{figure}

\end{itemize}

\pagebreak
\noindent{\Large Extra Sanity Checks}
\begin{itemize}
	\item Always return to the 7-step {\color{red} nodal analysis} if stuck or unsure.
	\item {\color{red} Care: Units!!!}
	\item The choice of ground does not matter. When labeling the currents through branches, the direction you pick does not matter.
	\item A Norton equivalent of a voltage source is not necessary, since a voltage source is a basic element. The Thevenin equivalent is just a voltage source with voltage $V_s$, that is, $R_{th}$ = 0.
	\item A Thevenin equivalent of a current source is not necessary because a current source is a basic element and cannot be represented as a voltage source. The Norton equivalent is just a current source with current $I_s$, that is, $R_{no}$ = $\infty$.
	\item Make sure $V_{DD} > V_{SS}$ (equal is not really useful...)
	\item We can connect an O-A to any other circuit, and the O-A will not disturb that circuit because it does not load the circuit (it is an open circuit). The output of the O-A can be connected to any other circuit (except a voltage source) to get the desired/expected voltage out of the O-A.
	\item $\Longrightarrow$ current source should not enter a terminal directly (parallel with a resistor and then enter, Norton-Thev. equiv. into a voltage source with a resistor).
	\item Positive gain $\iff$ non-inverting O-A; negative gain $\iff$ inverting O-A.
	\item Incident Matrix is the transpose of the matrix created for nodal analysis.
	\item Extra: Capacitors connected in series must have the same charge $Q$, and thus can calculate their voltage in proportion with $Q = CV$.
	\item Ground is a reference, the point we define as 0 in a circuit. There's only 1 GND, so we can connect all the grounds. But, we can't even connect ground to other nodes that happens to be at $0V$.
\end{itemize}

\end{document}