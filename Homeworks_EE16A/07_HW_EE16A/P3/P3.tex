\documentclass{article}
\usepackage[left=3cm, right=3cm, top=3cm]{geometry}
\usepackage{amssymb}
\usepackage{amsmath}
\usepackage{xcolor}
\usepackage{graphicx}
\begin{document}

{\LARGE 3. Fruity Fred} \\[.5cm]
{\Large (a) $R_{AB} = \rho\frac{L-kF}{A_c} + \rho\frac{L-kF}{A_c} = {\color{red} \frac{2\rho (L-kF)}{A_c}}$} \\[1cm]
{\Large (b) {\color{red} $F = \frac{A_cV_{out} + 2\rho L(V_{out}-1)}{2\rho k(V_{out}-1)}$}}

\begin{figure} [h!]
\begin{center}
	\includegraphics[width=0.75\linewidth]{/Users/Alan/Downloads/P2_b.jpg}
	\caption{Circuit Designed}
	\label{fig}
\end{center}
\end{figure}
{\color{white} a} \\

As deduced from the process on the picture,
so $R_{AB} = -\frac{V_{out}}{V_{out}-1}$. Also, as we derived from part (a), which gives $R_{AB} = \frac{2\rho (L-kF)}{A_c}$. Thus, we have that:
$$R_{AB} = \frac{2\rho (L-kF)}{A_c} = -\frac{V_{out}}{V_{out}-1}$$
$$\implies 2\rho (L-kF)\cdot-(V_{out}-1) = A_c\cdot V_{out}$$
$$\implies (2\rho L-2\rho kF)\cdot-(V_{out}-1) = A_c\cdot V_{out}$$
$$\implies -2\rho L(V_{out}-1) + 2\rho k(V_{out}-1)F = A_c\cdot V_{out}$$
$$\implies 2\rho k(V_{out}-1)F = A_c\cdot V_{out} + 2\rho L(V_{out}-1)$$
$$\implies F = \frac{A_cV_{out} + 2\rho L(V_{out}-1)}{2\rho k(V_{out}-1)}$$


\end{document}