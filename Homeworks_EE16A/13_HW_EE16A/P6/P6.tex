\documentclass{article}
\usepackage[left=3cm, right=3cm, top=3cm]{geometry}
\usepackage{amssymb}
\usepackage{amsmath}
\usepackage{mathrsfs}
\usepackage{xcolor}
\begin{document}

{\Large 6. OMP Exercise} \\[.5cm]
{\large {\color{red} (a) $x_1 = \frac{19}{3},\
 x_2 = 0,\
 x_3 = 0,\
 x_4 = \frac{8}{3}$ }} \\[.3cm]

Given that $\mathbf{M} = \begin{bmatrix}
	1 & 1 & 0 & -1 \\
	1 & 0 & -1 & 0 \\
	0 & 1 & 1 & 1
\end{bmatrix}$ \\[.3cm]

Iteration 1: \\

Calculating the four inner products ($\vec{b}$ with every column of $\mathbf{M}$) gives us $[10, 7, -3, -1]$, and the largest value is at column 1. Thus, we have
$\mathbf{A_1} = \begin{bmatrix}
    1 \\
 	1 \\
 	0
\end{bmatrix}$ and
$\vec{b_1} = \vec{b} = \begin{bmatrix}
	4 \\
	6 \\
 	3
\end{bmatrix}$.
So, we have the least squares solution
$\vec{x_1} =
(\mathbf{A_1}^T\mathbf{A_1})^{-1}\mathbf{A_1}^T\,\vec{b_1} =
\frac{1}{2}\cdot10 = 5$,
which gives us that $x_1 = 5$, and so we have our estimate currently at $\vec{r_1} = 5x_1$ with the residue after first iteration being
$\vec{y_1} =
\vec{b_1} - \mathbf{A_1}\vec{x_1} = \begin{bmatrix}
	-1 \\
	1 \\
	3
\end{bmatrix} \neq \vec{0}$. \\[.3cm]

Iteration 2: \\

Again, calculate the four inner products ($\vec{y_1}$ with every column of $\mathbf{M}$) gives us $[0, 2, 2, 4]$, and the largest value is at column 4. Thus, we have
$\mathbf{A_2} = \begin{bmatrix}
	1 & -1 \\
	1 & 0 \\
	0 & 1
\end{bmatrix}$ and
$\vec{b_2} = \vec{b} = \begin{bmatrix}
	4 \\
	6 \\
 	3
\end{bmatrix}$.
So, we can calculate the least squares solution
$\vec{x_2} = \begin{bmatrix}
	19/3 \\
	8/3
\end{bmatrix}$,
which gives us updated value that $x_1 = \frac{19}{3}$ and $x_4 = \frac{8}{3}$, and so the second residue is
$\vec{y_2} =
\vec{b_2} - \mathbf{A_2}\vec{x_2} = \begin{bmatrix}
	1/3 \\
	-1/3 \\
	1/3
\end{bmatrix}$ \\[.3cm]

Since we are given that $\vec{x}$ has only 2 non-zero entries, so we're done, with
{\color{red}
$$x_1 = \frac{19}{3},\
x_2 = 0,\
x_3 = 0,\
x_4 = \frac{8}{3}$$}

Therefore, we have that:
{\color{red}
\begin{center}
	$\vec{x} =
	\begin{bmatrix}
		19/3 \\
		0 \\
		0 \\
		8/3
	\end{bmatrix}$
\end{center} }

\noindent{\large {\color{red} (b)
$\vec{x} = \mathbf{A}\cdot\text{OMP}(\mathbf{M}\mathbf{A}, \vec{b})$ } } \\

Given matrix $\mathbf{A}$ and that $\vec{x} = \mathbf{A}\vec{x_a}$, we first calculate the inverse of $\mathbf{A},\ \mathbf{A}^{-1}$.
Then, by multiplying both sides of the equation by $\mathbf{A}^{-1}$, we have that $\vec{x_a} = \mathbf{A}^{-1}\mathbf{A}\vec{x_a} = \mathbf{A}^{-1}\vec{x}$. \\

Since we have that $\mathbf{M}\vec{x} = \vec{b}$, and since by definition, $\mathbf{A}\mathbf{A}^{-1} = \mathbf{I}$, so we have:
$$(\mathbf{M}\mathbf{A})\vec{x_a} =
\mathbf{M}\mathbf{A}\mathbf{A}^{-1}\vec{x} =
\mathbf{M}\vec{x} = \vec{b}$$

Now, since we are given that $\vec{x_a}$ is sparse, so our OMP function would successfully compute $\vec{x_a}$ by plugging in parameters $\mathbf{M}\mathbf{A}$ and $\vec{b}$, i.e. we can successfully determine
$\vec{x_a} = \text{OMP}(\mathbf{M}\mathbf{A}, \vec{b})$. \\

Finally, we can compute $\vec{x}$ back from $\vec{x_a}$ with the equation $\vec{x} = \mathbf{A}\vec{x_a}$. Therefore, in other words, we could compute $\vec{x}$ in this method:
$$\vec{x} = \mathbf{A}\cdot\text{OMP}(\mathbf{M}\mathbf{A}, \vec{b})$$

\end{document}